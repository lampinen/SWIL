\documentclass[11pt]{article}

\usepackage{subcaption}
\usepackage{graphicx}
\usepackage{amsmath}
\usepackage{relsize}
\usepackage{natbib}
\usepackage{float}
\usepackage[margin=1in]{geometry}

\floatstyle{boxed}
\restylefloat{figure}
\setlength{\parindent}{0em}
\setlength{\parskip}{1em}
\begin{document}

Suppose we have a hierarchically structured dataset generated by a tree with $N$ splits ($N+1$ depths of nodes) and $C$ children per node, according to the following model: 
\begin{itemize}
\item assign $f$ binary features to be one for all items
\item Traverse down the tree. At each node, create $f$ binary features for each of the $C$ children that are one for descendants of that child, and 0 otherwise. 
\end{itemize}
What is the singular value of each mode of the SVD in this case? \par
\textbf{Solution:} There are $N+1$ unique singular values of the SVD, with the $k$-th unique singular value having multiplicity 
$$m_k = \begin{cases}
1 & k = 1 \\
(C-1) C^{k-2}  & k > 1
\end{cases}
$$
because $C^{k-2}$ groups were created at higher levels, and each split of one of these creates $fC$ features but only $C-1$ linearly independent modes. \par
Furthermore, the square of the $k$-th singular value, i.e. the variance explained by each of those $m_k$ modes is:
$$s_k^2 = FC^{N-k+2} \sum_{i=0}^{N-k+2} \frac{1}{C^i}$$
This is because there are $F$ features per group, there are $R = C^{N-k+2}$ items affected by a $k$-th mode, and there are $N-k$ levels below that, at which the predictive value per feature per row (i.e. one entry in the outer product) at the $i$-th level below $k$ is $1/C^{i+1}$. This is squared to calculate the variance, but then one factor of $C^{i+1}$ cancels with the $C^{i+1}$ columns, thus yielding the formula above. \par
Where does each entry in the outer product come from? Note that within each of the $i$ remaining sections of the data, each row has $F$ ones distributed over $FC^{i+1}$ columns. The ratio of these gives $1/C^{i+1}$. 

\end{document}
